% !TEX encoding = UTF-8 Unicode
% -*- coding: UTF-8; -*-
% vim: set fenc=utf-8
\section{Ход работы}
\subsection{Подходы к тестированию}
В ходе работы были изучены следующие подходы к тестированию:
\begin{itemize}
    \item функциональное тестирование;
    \item нагрузочное тестирование;
    \item стресс-тестирование;
    \item компонентное тестирование;
    \item интеграционное тестирование;
    \item smoke-тестирование;
    \item регрессионное тестирование.
\end{itemize}

Функциональное тестирование -- это тестирование ПО в целях проверки реализуемости функциональных требований, то есть способности ПО в определённых условиях решать задачи, нужные пользователям. Функциональные требования определяют, что именно делает ПО, какие задачи оно решает.\par

Нагрузочное тестирование -- подвид тестирования производительности, сбор показателей и определение производительности и времени отклика программно-технической системы или устройства в ответ на внешний запрос с целью установления соответствия требованиям, предъявляемым к данной системе (устройству).\par

Стресс-тестирование -- один из видов тестирования программного обеспечения, которое оценивает надёжность и устойчивость системы в условиях превышения пределов нормального функционирования. Стресс-тестирование особенно необходимо для «критически важного» ПО, однако также используется и для остального ПО.\par

Компонентное тестирование -- процесс в программировании, позволяющий проверить на корректность работы отдельные компоненты составной системы.\par

Интеграционное тестирование -- одна из фаз тестирования программного обеспечения, при которой отдельные программные модули объединяются и тестируются в группе. Обычно интеграционное тестирование проводится после компонентного тестирования.\par

Smoke-тестирование в тестировании программного обеспечения означает минимальный набор тестов на явные ошибки. <<Дымовой тест>> обычно выполняется самим программистом. Не прошедшую этот тест программу не имеет смысла отдавать на более глубокое тестирование.\par

Регрессионное тестирование -- собирательное название для всех видов тестирования программного обеспечения, направленных на обнаружение ошибок в уже протестированных участках исходного кода. Такие ошибки -- когда после внесения изменений в программу перестаёт работать то, что должно было продолжать работать, -- называют регрессионными ошибками.\par

Разработанное в ходе практики программное обеспечение относится к функциональным тестам, и основано на фреймворке py.test.\\

\subsection{Py.test}
Py.test -- это набор библиотек для написания тестов на языке программирования Python. Задание практики было выполнено с использованием данного фреймворка.\par

Преимущества:
\begin{itemize}
    \item независимость от API;
    \item подробный отчет, в том числе выгрузка в JUnitXML (для интеграции с Jenkins);
    \item сам вид отчета может изменяться (включая цвета) дополнительными модулями;
    \item удобный assert (стандартный из Python);
    \item динамические фикстуры всех уровней, которые могут вызываться как автоматически, так и для конкретных тестов;
    \item дополнительные возможности фикстур (возвращаемое значение, финализаторы, область видимости, объект request, автоиспользование, вложенные фикстуры);
    \item параметризация тестов, то есть запуск одного и того же теста с разными наборами параметров;
    \item метки (marks), позволяющие пропустить любой тест, пометить тест, как падающий (и это его ожидаемое поведение, что полезно при разработке) или просто именовать набор тестов, чтобы можно было запускать только его по имени;
    \item данный модуль имеет достаточно большой список дополнительных модулей, которые можно установить отдельно;
    \item возможность запуска тестов написанных на unittest и nose, то есть полная обратная совместимость с ними.
\end{itemize}

Недостатки:
\begin{itemize}
    \item отсутствие дополнительного уровня вложенности;
    \item необходимость отдельной установки модуля;
    \item для использования PyTest требуется немного больше знаний Python, чем для того же unittest.\\
\end{itemize}

\subsection{Описание программы}

Разработанное программное обеспечение направлено на проверку соответствия фактических прав, предоставляемых пользователям, эталонным правам, указанным в конфигурационном файле, при выполнении запросов к httpapi.\par

В конфигурационном файле описываются следующие сущности:
\begin{itemize}
    \item роли -- группы пользователей, наделённые определёнными правами;
    \item пользователи -- учётные записи, используемые для входа в систему, имеющие одну или несколько ролей;
    \item методы -- url'ы httpapi снабженные описанием и указанием вида http-запроса;
    \item ограничения -- перечень ролей, уполномоченных выполнять определённый запрос, или же <<anyone>>, в случае, если выполнение запроса доступно любому авторизованному пользователю.
\end{itemize}

Перед началом работы программа считывает конфигурационный файл, и проводит параметризацию тестов, т.е. подготавливает наборы входных данных и ожидаемых выходных данных. После чего для каждого набора входных данных осуществляется запрос к httpapi на авторизацию пользователя (в случае если указанный во входных данных пользователь не является анонимным), а затем и сам запрос, права выполнения которого необходимо проверить. В зависимости от ответа веб-сервера принимается решение о соответствии полученных прав ожидаемым. Любое отклонение от эталонных прав, указанных в конфигурационном файле, будь то превышение полномочий, или недостаток прав, считается ошибкой, и приводит к падению теста.\par

Результаты выполнения тестов выводятся в консоль, и могут быть использованы в системах Continious Integration. Разработанное программное обеспечение включено в набор автотестов PT ISIM, исполняемых на билд-серверах непрерывной интеграции TeamCity.\\

