% !TEX encoding = UTF-8 Unicode
% -*- coding: UTF-8; -*-
% vim: set fenc=utf-8
\section{Характеристика предприятия}

Компания Positive Technologies за 15 лет существования завоевала лидирующие позиции на отечественном и европейском рынке систем анализа защищенности и соответствия стандартам, а также защиты веб-приложений. Специалисты компании заслужили репутацию экспертов международного уровня в вопросах защиты SCADA- и ERP-систем, банков и телекомов. Производимое программное обеспечение для защиты критически важных объектов сертифицировано Минобороны России, ФСБ России и ФСТЭК России, а также успешно зарекомендовало себя и за рубежом – в Европе и Азии.\par

Продукты и услуги Positive Technologies обеспечивают:
\begin{itemize}
    \item анализ защищенности и оценку соответствия стандартам;
    \item мониторинг событий безопасности и предотвращение вторжений;
    \item блокирование атак, включая ранее неизвестные (0-day);
    \item расследование инцидентов и оценку защитных мер;
    \item анализ безопасности кода приложений и построение безопасной разработки.
\end{itemize}

Positive Technologies сегодня -- это международная компания с несколькими представительствами и R\&D центрами в России и по всему миру, в том числе в Великобритании, Чехии и других странах. В компании работает более 250 экспертов по защите ERP, SCADA, банков и телекомов, веб- и мобильных приложений. Линейка успешно работающих продуктов компании наилучшим образом адаптирована для бизнес-клиентов в различных отраслях и странах.\par

Репутация экспертов мирового уровня по вопросам защиты самых разнообразных устройств и инфраструктур подтверждена обширным списком партнеров и клиентов:
\begin{itemize}
    \item работа более чем с 1000 компаний в 30 странах мира;
    \item сотрудничество с Cisco, Google, Huawei, Microsoft, Oracle, SAP, Siemens, Schneider Electric, Honeywell и другими крупными мировыми вендорами;
    \item высокие оценки международных аналитических агентств. Так, в 2015 и 2016 году компания названа <<визионером>> в исследовании Gartner Magic Quadrant по системам защиты веб-приложений (WAF);
    \item Positive Technologies несколько лет подряд проводит собственный научно- практический форум Positive Hack Days, в работе которого ежегодно участвуют более 3000 экспертов. В рамках конференции проводятся сотни докладов и мастер-классов по самым острым темам ИБ, а также практические конкурсы по анализу защищенности промышленных систем управления, банковских сервисов, мобильной связи и веб-приложений;
    \item компания аккумулирует самые свежие новости индустрии на портале SecurityLab.ru;
    \item компания разрабатывает образовательные программы для ведущих вузов страны и помогает растить специалистов со «студенческой скамьи» - в рамках образовательной программы Positive Education учебные материалы, подготовленные экспертами компании, используют в своих курсах уже более 50 российских вузов.
\end{itemize}

Все решения Positive Technologies проектируются с учётом большого опыта защиты бизнеса в различных отраслях, а также специфики требований регуляторов. Специалисты компании участвуют в работе технических комитетов Росстандарта и рабочих групп ФСТЭК, оказывая экспертную помощь в формировании требований безопасности. Производимые продукты максимально соответствуют российским и международным стандартам безопасности, включая стандарты PCI DSS и ЦБ РС БР ИББС-2.6-2014, приказы ФСТЭК № 17 и 21.\par

MaxPatrol 8 -- система контроля защищенности и соответствия стандартам. Механизмы тестирования на проникновение (Pentest), системных проверок (Audit) и контроля соответствия стандартам (Compliance) в сочетании с поддержкой анализа различных операционных систем, СУБД и web-приложений обеспечивают непрерывный мониторинг безопасности на всех уровнях информационной системы. Регулярно обновляемая база знаний об уязвимостях - одна из крупнейших в мире.\par

MaxPatrol SIEM -- инновационное решение класса SIEM для управления событиями и информацией ИБ с целью выявления инцидентов в режиме реального времени. MaxPatrol SIEM предлагает механизм передачи экспертизы информационной безопасности напрямую в продукт и позволяет получить эффективную SIEM-систему даже с  минимальными ресурсами эксплуатации. MaxPatrol SIEM является ключевым элементом новой платформы средств безопасности Positive Technologies, в основе которой лежит сбор и анализ информации обо всех активах и событиях защищаемой системы.\par

PT Application Firewall -- PT Application Firewall -- современное решение для защиты веб-приложений от известных и неизвестных атак, включая OWASP Top 10, автоматизированные атаки, атаки на стороне клиента и атаки нулевого дня. Решение основано на передовых технологиях и регулярных исследованиях экспертов, чтобы обеспечить высочайший уровень безопасности и непрерывность бизнес-процессов организации.\par

PT Application Inspector -- анализатор защищенности исходного кода приложений. За счёт комбинации статических, динамических и интерактивных методов анализа имеет очень низкий уровень ложных срабатываний, радикально снижает затраты на ручную проверку результатов.\par

PT MultiScanner -- система выявления вредоносных файлов и ссылок PT MultiScanner позволяет значительно повысить точность и оперативность обнаружения угроз за счет многопоточного сканирования несколькими антивирусными ядрами, в сочетании с другими методами выявления ВПО.\par

XSpider -- основная задача сканера XSpider -- обнаружить уязвимости в сетевых ресурсах до того, как это будет сделано злоумышленниками, а также выдать чёткие и понятные рекомендации по устранению обнаруженных уязвимостей.\par

PT ISIM -- PT Industrial Security Incident Manager -- это система управления инцидентами кибербезопасности АСУ ТП, которая выявляет хакерские атаки и помогает в расследовании инцидентов на критически важных объектах. Не влияя на технологический процесс, PT ISIM параллельно с ним анализирует копию сетевого трафика, выявляет взаимосвязи между событиями безопасности и наглядно визуализирует потенциальные атаки на топологии сети и схеме промышленного объекта.\\
