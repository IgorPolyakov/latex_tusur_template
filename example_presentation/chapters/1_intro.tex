% !TEX encoding = UTF-8 Unicode
% -*- coding: UTF-8; -*-
% vim: set fenc=utf-8
\section{Введение}
Автоматизированные системы управления технологическими процессами (АСУ ТП) сегодня применяются во множестве отраслей -- в нефтегазовой промышленности, металлургии, энергетике, космонавтике, медицине. Традиционно при проектировании АСУ ТП исходят из предположения, что подобные системы -- это часть замкнутой экосистемы, которая рассчитана на различные режимы работы, включая аварийные, что позволяет в определённой степени пренебрегать рисками информационной безопасности.\par

\begin{figure}[h!]
    \centering
    \includegraphics[width=1\textwidth]{01}
    \caption{Количество компонентов АСУ ТП, доступных в сети Интернет (распределение по странам)}
    \label{img:01}
\end{figure}

Существуют ли подобные условия в действительности? Многочисленные исследования кибербезопасности АСУ ТП доказывают, что нет. Миф об АСУ ТП, функционирующих внутри некой доверенной зоны, перестал существовать вместе с понятием <<воздушный зазор>> (физическая изоляция технологической сети). Как правило, именно воздушный зазор считался действенным средством против инцидентов информационной безопасности в промышленности.\par

Для эффективного решения бизнес-задач промышленные компании, напротив, стремятся интегрировать корпоративные и производственные IТ-инфраструктуры. Все чаще технологические сети намеренно или по ошибке подключают к публичным сетям, тем самым ставя под угрозу их безопасность. По состоянию на начало 2017 года эксперты Positive Technologies выявили более 160 000 различных компонентов АСУ ТП, напрямую подключенных к сети Интернет.\par

Еще одна проблема -- уязвимости подсистем и компонентов АСУ ТП. Их количество уже давно остается стабильно высоким. Как показали исследования, большая часть уязвимостей, опубликованных в 2016 году, приходится на устройства, выполняющие функции диспетчеризации и мониторинга (ЧМИ/SCADA), а наиболее распространенные типы уязвимостей -- <<Удаленное выполнение кода>>, <<Отказ в обслуживании>> и <<Раскрытие информации>>.\par

\begin{figure}[h!]
    \centering
    \includegraphics[width=1\textwidth]{02}
    \caption{Распространенные типы уязвимостей компонентов АСУ ТП}
    \label{img:02}
\end{figure}

При этом большинство уязвимостей могут быть проэксплуатированы удаленно злоумышленником низкой квалификации, а устранение уязвимостей зачастую попросту невозможно по различным объективным причинам. Таким образом, внешний нарушитель, проникнув в технологическую сеть через корпоративные или публичные сети, может сразу получить максимальные возможности по нарушению работы производственной системы.\par

Кроме того, к инцидентам информационной безопасности приводят и действия персонала предприятия. Причина может быть в низкой квалификации, халатности, несоблюдении регламентов и правил доступа. Простые пароли, записанные на бумаге, несанкционированное подключение электронных носителей и устройств (USB-накопителей, смартфонов, GSM-модемов) к АРМ оператора, проникновение вредоносного ПО из корпоративной сети (например, через электронную почту) -- это лишь малая часть событий, приводящих к инцидентам и нарушениям в работе технологических систем.\par

Отдельно стоит отметить угрозы, связанные с персоналом подрядчиков, принимающих участие в проектировании, построении и обслуживании АСУ ТП. Как правило, таким специалистам предоставляются максимальные системные привилегии, а также полный физический или удаленный доступ, при этом контролировать их действия по разным причинам затруднительно. Отсюда случаи некорректной настройки оборудования и злоумышленного изменения режимных параметров, заражения АРМ вредоносными программами в ходе регламентного и оперативного обслуживания.\par

\begin{figure}[h!]
    \centering
    \includegraphics[width=1\textwidth]{03}
    \caption{Уязвимости по основным производителям компонентов АСУ ТП}
    \label{img:03}
\end{figure}

Обеспечение безопасности АСУ ТП от подобных угроз требует комплексного подхода, включающего физическую сегментацию сетей, защиту периметра, внедрение политик безопасности и постоянный мониторинг защищенности. Поскольку стандартные средства защиты узлов и периметра не дают полного представления о текущем состоянии защищенности, рекомендуется использовать специализированные комплексы для непрерывного анализа защищенности АСУ ТП и детектирования инцидентов в реальном времени. Анализ политик разграничения доступа одного из таких комплексов -- Positive Technologies Industrial Security Incident Manager и являлся предметом практики.\\
