% !TEX encoding = UTF-8 Unicode
% -*- coding: UTF-8; -*-
% vim: set fenc=utf-8
\begin{center}
 Министерство образования и науки Российской Федерации\\
 Федеральное государственное бюджетное образовательное учреждение высшего образования\\
 <<ТОМСКИЙ ГОСУДАРСТВЕННЫЙ УНИВЕРСИТЕТ СИСТЕМ УПРАВЛЕНИЯ И РАДИОЭЛЕКТРОНИКИ>> (ТУСУР)\\
 Кафедра комплексной информационной безопасности электронно-вычислительных систем (КИБЭВС)\\
\end{center}

\vspace{2em}

\begin{center}
ЗАДАНИЕ\\
на производственную практику
\end{center}

Студенту Иванову Ивану Ивановичу\par
группы 777 факультета Безопасности\par
\begin{enumerate}
    \item Тема задания:
    Разработка программного обеспечения для проверки политик разграничения доступа к программным интерфейсам.
    \item Исходные данные:
    Задание на производственную практику.
    \item Перечень вопросов, подлежащих разработке:
    изучение системы управления инцидентами кибербезопасности АСУ ТП PT ISIM, установка и настройка системы PT ISIM, анализ конфигурационных файлов системы PT ISIM, выработка единого формата описания политик разграничения доступа к программным интерфейсам системы PT ISIM, разработка программного обеспечения для проверки политик разграничения доступа к программным интерфейсам системы PT ISIM, анализ результатов работы разработанного программного обеспечения, внедрение разработанного программного обеспечения в существующую систему автотестов, написание отчёта.
\end{enumerate}

\vspace{2em}


Руководитель практики от предприятия\par
Солдатов С.Е., старший инженер по качеству, АО Позитив Текнолоджиз\underline{\hspace{2cm}}\par
Задание принял к исполнению\underline{\hspace{2cm}}\par
<<\underline{\hspace{1cm}}>>\underline{\hspace{3cm}}2017г.\par
