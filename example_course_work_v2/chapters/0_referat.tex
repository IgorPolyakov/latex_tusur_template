\begin{center}
РЕФЕРАТ
\end{center}

Отчет содержит 32 страницы, 9 рисунков, 22 источника, 1 приложение.\par 

CTF, ATTACK-DEFENSE, СОРЕВНОВАНИЯ, KEVA, ЗАЩИТА ИНФОРМАЦИИ,  GIT, PYTHON, C, ZABBIX, DPDK, ТЕСТИРОВОЧНЫЙ СТЕНД, СЕТЕВОЙ ТРАФИК, ЗЕРКАЛИРОВАНИЕ ТРАФИКА, СЕТЕВАЯ КАРТА, НЕЙРОННЫЕ СЕТИ, АНАЛИЗ ТРАФИКА.\par 

Объект исследования и разработки --- сбор сетевого трафика.\par 

Цель работы — решение проблем мониторинга сетевого трафика при высокой загрузке сети на соревнованях по информационной безопасности, путём создания тестового стенда для зеркалирования трафика с использованием фреймворка DPDK, и установки и настройки системы мониторинга сетевого оборудования Zabbix.\par

Результаты работы в данном семестре:
\begin{itemize}
\item проведено исследование методов для анализа трафика;
\item проведено ознакомление со значительным количеством сетевых утилит и программ для захвата трафика;
\item изучено устройство и принципы работы с набором библиотек DPDK;
\item разработаны требования к тестовому стенду для прозрачного зеркалирования трафика;
\item спроектирован и подготовлен тестовый стенд для дальнейшего использования на соревнованиях SibirCTF;
\item произведено тестирование стенда, в том числе в условиях, приближенных к реальным;
\item был установлен и настроен Zabbix-сервер для мониторинга нагрузки и производительности сетевого оборудования инфраструктуры соревнований.
\end{itemize} 

В качестве инструментария для выполнения данной работы были использованы: система контроля версий Git, система мониторинга компьютерной сети Zabbix, языки программирования C, Python, фреймворк DPDK.\par

Пояснительная записка выполнена при помощи системы компьютерной вёрстки \LaTeX.

\clearpage

\begin{center}
THE ABSTRACT
\end{center}

Course work contains 32 pages, 9 pictures, 22 references, 1 attachment.\par

CTF, ATTACK-DEFENSE, COMPETITIONS, KEVA, INFORMATION SECURITY, GIT, PYTHON, C, ZABBIX,
DPDK, DEMOSTAND, NETWORK TRAFFIC, TRAFFIC MIRRO- RING, NETWORK CARD, NEURAL NETWORKS, TRAFFIC ANALYSIS.\par

The study object is network traffic grabbing.\par

The purpose of the work is solving network traffic monitoring problems in high-load networks on information security competitions by creation of the demostand for traffic mirroring via DPDK framework and Zabbix monitoring system installation and configuring.\par

This term results:

\begin{itemize}
    \item performed research of traffic analysis methods;
    \item performed research of network traffic grabbing utilities and programs;
    \item performed research of DPDK framework methods;
    \item developed demostand requirements for hidden traffic mirroring;
    \item designed and prepared demostand for using on SibirCTF;
    \item performed demostand testing (including environment closed to real);
    \item installed and configured Zabbix server for payload and perfomance monitoring of network devices.
\end{itemize}

This work is made with: version control system Git, Zabbix network monitor, C and Python programming language, DPDK framework.\par

Course work is made with \LaTeX.\par

\clearpage