\section{Введение}
\subsection{Краткая теория}

CTF (Capture the flag с англ. <<Захват флага>>) --- это игра, в которой несколько команд соревнуются друг с другом. В игре проверяются способности специалистов защитить сложную незнакомую систему с сохранением необходимой функциональности. \par 

По типу, соревнования делятся на два типа: task-based (квесты), attack-defense (классические соревнования).\par 

В случае соревнований task-based (или jeopardy) игрокам предоставляется набор тасков (заданий), к которым требуется найти ответ и отправить его. Ответ представляет собой флаг: это может быть набор символов или произвольная фраза. За верно выполненное задание команда получает определенное количество очков. Чем задание сложнее, тем больше очков будет полагаться за правильный ответ. Все задания в CTF-соревнованиях формата task-based можно разделить на несколько категорий: например, это задачи на администрирование, криптографию и стеганографию, задачи на нахождение веб-уязвимостей и любимые многими задания категории joy – развлекательные задачи разнообразной тематики.\par

Задача, с которой сталкиваются участники классического типа соревнований (attack-defense), сходна с реальной работой консультанта по информационной безопасности в новой организации. Необходимость защищать свой сервер и, в то же время, исследовать, атаковать и размещать свои флаги на чужих серверах делает соревнование более динамичным и зрелищным.\par 

Команды получают идентичные образы. В ходе игры жюри будет размещать на серверах команд некоторую информацию (флаги) и проверять их доступность. Задача команд – обнаружить как можно большее количество флагов противника, при этом не давая обнаружить свои. За найденные флаги команда получает очки за нападение, за утерянные – ничего. За поддержание сервисов в рабочем и исправном состоянии команда получает очки за защиту, соответственно, за нерабочий сервис – ничего. Команда победитель определяется путем подсчета баллов по окончанию игры, победителем считается та команда, которая будет иметь большее количество баллов.\\

\subsection{Назначение и область применения}
Разрабатываемый программно-аппаратный комплекс предназначен для внедрения в сетевую инфраструктуру соревнований в области информационной безопасности Capture The Flag формата Attack-Defense, с целью сбора игрового трафика, и мониторинга состояния сети, и оборудования в ней.

\clearpage

\subsection{Постановка проблемы}
Основной задачей группы ГПО КИБЭВС-1502 в прошлом семестре была разработка программно-аппаратного комплекса для проведения соревнований в области информационной безопасности. В рамках данной задачи осенью 2016 года были проведены III ежегодные межвузовские межрегиональные соревнования по защите информации SibirCTF.\par 

Во время проведения соревнований команда организаторов столкнулась с рядом сложностей.\par

1.3.1  Проблема маршрутизации, связанные со спецификой соревнований, обязательным требованием которых является наличие <<маскарада>> -- средства сокрытия оригинального источника входящего трафика от участников команд. При большом объёме трафика, появляющегося во второй половине игры, из-за активного использования участниками автоматизированных средств эксплуатации уязвимостей, модуль <<маскарада>> автоматически отключается, как неспособный обработать весь объём трафика, для обеспечения продолжения функционирования базовых функций маршрутизатора.\par

1.3.2  Проблема сбора и хранения трафика регулярно возникает на соревнованиях CTF формата attack-defense начиная с момента их появления. Особенно остро данная проблема встала на соревнованиях SibirCTF2016. Сбор и последующее хранение трафика игровой сети необходимо осуществлять по ряду причин, среди которых:

\begin{itemize}
\item необходимость наличия копии трафика для вынесения жюри аргументированного решения в случае подозрения противоречащих правилам соревнований действий со стороны участников;
\item необходимость последующего анализа трафика для повышения профессиональных качеств команды организаторов;
\item необходимость наличия копии трафика с угрозами информационной безопасности для последующего использования в качестве обучающего материала для разрабатываемых интеллектуальных систем связанных с информационной безопасностью.
\end{itemize}\par 

1.3.3  Проблема мониторинга нагрузки, производительности, и используемых ресурсов оборудования игровой инфраструктуры. В целях оперативного реагирования организаторами на нештатные ситуации, возникающие в процессе эксплуатации участниками команд игровой сети, необходимо постоянно отслеживать в реальном времени потребление ресурсов, текущую нагрузку, а также другие параметры используемого оборудования.\\

\subsection{Основное направление на данный семестр}

По итогам прошлого семестра ГПО был разработан программно-аппаратный комплекс для проведения соревнований по информационной безопасности, включающий в себя:

\begin{itemize}
\item ядро программно-аппаратного комплекса - жюрейную систему;
\item сетевую инфраструктуру для проведения соревнований, состоящую из доступного команде организаторов сетевого оборудования.
\end{itemize}

Таким образом, основным направлением работы на данный семестр стало решение проблемы сбора и хранения игрового трафика, а также решение вопросов мониторинга.\\
\clearpage