\section*{Список использованных источников}

\addcontentsline{toc}{section}{Список использованных источников}

\begin{itemize}
\item Что такое Intel DPDK? [Электронный ресурс] / SDNBLOGGER // SDNBLOG : портал. – Опубл.: 28.03.2016. – URL: https://sdnblog.ru/what-is-intel-dpdk, свободный. – Загл. с экрана (дата обращения: 22.05.2017).
\item Ярош, Ю. Использование DPDK для обеспечения высокой производительности прикладных решений (часть 0) [Электронный ресурс] // Хабрахабр : сайт. – Опубл.: 10.12.2015. –URL: https://habrahabr.ru/post/267591/, свободный. – Загл. с экрана (дата обр.: 22.05.2017).
\item Programmer's Guide : [Electronic resource] // DPDK: Data Plane Development Kit : [website]. San Francisco, 2013–2017. – URL: http://dpdk.org/doc/guides/prog\_ guide/, free. – Tit. screen (usage date: 22.05.2017). 
\item Сердюк, В. Вы атакованы – защищайтесь (методология выявления атак) [Электронный ресурс] // Диалог Наука : сайт. – М., 1998–2017. – URL: https://www.dialognauka.ru/ \\press-center/article/4770, свободный. – Загл. с экрана (дата обращения: 22.05.2017). 
\item Scott Chacon. Pro Git: professional version control. 2011. \\ URL: http://progit.org/ebook/progit.pdf.
\item С.М. Львовский. Набор и вёрстка в системе L A TEX. МЦНМО, 2006. С. 448.
\item И. А. Чеботаев, П. З. Котельников. L A TEX 2 по-русски. Сибирский Хронограф, 2004. 489 с.
\item  Python. [Электронный ресурс] // ru.wikipedia.org:[сайт]. 2015.\\ URL: https://ru.wikipedia.org/wiki/Python.
\item Зенов, А. Ю. Применение нейросетевых алгоритмов в системах охраны периметра / А. Ю. Зенов, Н. В. Мясникова // Изв. вузов. Поволж. регион. Техн. науки. – 2012. – № 3. – С. 15–24.
\item Исаев, С. Популярно о генетических алгоритмах [Электронный ресурс] // \\ Algolist.manual.ru: Алгоритмы. Методы. Исходники : сайт. – 2000–2017. – URL: \\http://algolist.manual.ru/ai/ga/ga1.php\# ga , свободный. – Загл. с экрана (дата обращения: 22.05.2017).
\item Генетические алгоритмы. От теории к практике [Электронный ресурс] / knok16 // Хабрахабр : сайт. – Опубл.: 13.02.2012. –URL: https://habrahabr.ru/post/138091, свободный. – Загл. с экрана (дата обращения: 22.05.2017).
\item Генетический алгоритм. Просто о сложном [Электронный ресурс] / Марк@mrk-andreev // Хабрахабр : сайт. – Опубл.: 20.09.2011. – URL: https://habrahabr.ru/post/128704, свободный. – Загл. с экрана (дата обращения: 22.05.2017).
\item Камаев, В. А. Методология обнаружения вторжений / В. А. Камаев, В. В. Натров // Изв. Волгогр. техн. ун-та. – 2006. – № 4. – С. 148–153.
\item Введение в нечеткие системы [Электронный ресурс]. – URL: \\http://www.igce.comcor.ru/AI\_ mag/NN/FuzzySystems/FuzzySystems.html, свободный. – Загл. с экрана (дата обращения: 22.05.2017).
\item Нейронные сети [Электронный ресурс] // AIPortal : портал. – 2009–2017. – URL: http://www.aiportal.ru/articles/neural-networks/neural-networks.html, свободный. – Загл. с экрана (дата обращения: 22.05.2017).
\item Большев, А. К. Применение нейронных сетей для обнаружения вторжений в компьютерные сети /, А. К. Большев, В. В. Яновский // Вестн. СПбГУ. Сер. 10, Приклад. математика:Информатика. Процессы упр. – 2010. – Вып. 1. – С. 129–135. 
\item Scholz D. A Look at Intel’s Dataplane Development Kit [Electronic resource] // Future Internet (FI) and Innovative Internet Technologies and Mobile Communications (IITM) : Proc. of the Seminars, Summer Semester / Eds. Georg Carle, Daniel Raumer, Lukas Schwaighofer. – Munich, 2014. – Vol. NET-2014-08-1. – P. 115–123. – doi: 10.2313/NET-2014-08-1\_ 15
\item Жигулин, П. В. Анализ сетевого трафика с помощью нейронных сетей [Электронный ресурс] / П. В. Жигулин, Д. Э. Подворчан // ТУСУР : информ. портал. – URL: http://storage.tusur.ru/files/425/КИБЭВС-1005\_ Жигулин\_ П.В.\_\_ Подворчан\_ Д.Э.pdf
\item Мустафаев, А. Г. Нейросетевая система обнаружения компьютерных атак на основе анализа сетевого трафика // Вопр. безопасности. – 2016. – № 2. – С. 1–7. 
\item Сухов, В. Е. Система обнаружения аномалий сетевого трафика на основе искусственных иммунных систем и нейросетевых детекторов // Вестн. РГРТУ. – 2015. – № 54, ч. 1. – С. 84–90.
\item Басараб, М. А. Анализ сетевого трафика корпоративной сети университета методами нелинейной динамики [Электронный ресурс] / М. А. Басараб, А. В. Колесников, И. П. Иванов // Наука и образование. – 2013. – № 8. – С. 341–352. – doi: 10.7463/0813.0587054
\item Гончаров, В. А. Исследование возможностей противодействия сетевым информационным атакам со стороны защищенных ос и систем обнаружения информационных атак / В. А. Гончаров, В. Н. Пржегорлинский // Вестн. РГРТА. 2007. – Вып. 20. – С. 10–14.
\end{itemize}
